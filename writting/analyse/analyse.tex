\documentclass[onecolumn, 12pt]{article}

%!TeX spellcheck =fr
  \usepackage[utf8]{inputenc}
  \usepackage{hyperref}
  \usepackage[T1]{fontenc}
  \usepackage[francais]{babel}
  \usepackage{layout}
  \usepackage{verbatim}[4]
  \usepackage{color}

  \usepackage{graphicx}
  \usepackage{caption}
  \usepackage{subcaption}

  \usepackage{titlesec}

  \setcounter{secnumdepth}{4}

  \titleformat{\paragraph}
  {\normalfont\normalsize\bfseries}{\theparagraph}{1em}{}
  \titlespacing*{\paragraph}
  {0pt}{3.25ex plus 1ex minus .2ex}{1.5ex plus .2ex}


\title{Analyse des résultats FNC}
  \author{Enzo Poggio}
\begin{document}
\maketitle{}


\begin{abstract}
 Dans cette première partie nous tentons de faire une analyse d'erreurs
 comparative des participants de la FNC. Nous avons reproduit les résultats de
 tous les participants\footnote{À l'exeption de Solat in the Swen qui ont laissé
  une copie de leur CSV de leurs résultats, pour leurs deux modèles sur leur
 répertoire Github.}.
\end{abstract}
\section{Un aperçu de l'ensemble de test.}
\begin{center}
 
 \begin{tabular}{ r | c c c c | c}
  Stance      & unrelated & agree  & disagree & discuss &          
  Somme    \\
  \hline
  Nombre      & 18349     & 1903   & 697      & 4464    & 25413    \\
  Pourcentage & 72.20\%   & 7.48\% & 2.74\%   & 17.56\% & 100\%    \\
  Score FNC   & 4587.25   & 1903   & 697      & 4464    & 11651.25 \\
 \end{tabular}
 \captionof{table}{Répartiotion des données dans l'ensembe de test.}
\end{center}

Il y a 7064 points à marquer avec les \textit{Related} et seulement 4587.25
points avec les \textit{Unrelated}.
Ainsi comme nous l'avions vu précédement trouvé
un \textit{Related} rapporte 4 fois plus de points qe trouvé un \textit{Unrelated}.

Afin de produire des hypothèses non-biaisées  \footnote{Du moins des hypothèses qui ne soient pas ajustées totalement à l'ensemble de test final.}, nous allons faire des observations des résultats des différents modèles sur 80\% des entrées de l'ensemble de test.

\begin{center}
 \begin{tabular}{ r | c c c c | c}
  Stance      & unrelated & agree  & disagree & discuss &        
  Somme    \\
  \hline
  Nombre      & 14662     & 1568   & 550      & 3550    & 20330  \\
  Pourcentage & 72.12\%   & 7.71\% & 2.70\%   & 17.46\% & 100\%  \\
  Score FNC   & 3665.5    & 1568   & 550      & 3550    & 9333.5 \\
 \end{tabular}
 \captionof{table}{Répartition des données dans l'ensembe de test à 80\%.}
\end{center}

Le corpus de test étant partiellement ordonné nous avons retiré les 20\% de manière ordonnée\footnote{En effet, nous avons retiré toutes les entrées dont l'indice été divisible par 5 ou 10} aussi.

\section{Les résultats et les scores des participants.}
Ici nous présentons les différents résultats des modèles de chaque participant. Notez que les tables de confusions ont horizontalement les labels de vérité et verticalement les prédictions.
\subsection{Solat in the Swen.}
\begin{center}
 \begin{tabular}{ r | c c c c | c }
  \multicolumn{6}{c}{solat}\\
  \hline
            & agree & disagree & discuss & unrelated & Somme \\
  \hline
  agree     & 927   & 11       & 478     & 152       & 1568  \\
  disagree  & 217   & 8        & 235     & 90        & 550   \\
  discuss   & 661   & 5        & 2700    & 184       & 3550  \\
  unrelated & 26    & 0        & 172     & 14464     & 14662 \\
  \hline
  Somme     & 1831  & 24       & 3585    & 14890     & 20330 \\
 \end{tabular}
 \captionof{table}{}
\end{center}

On remarque que la classe \textbf{disagree} est largement sous-représentée. Cette classe est aussi sous-représentée dans l'ensemble d'entrainement. Ce qui explique potentiellement pourquoi nous ne détectons pas ces traits distinctifs. Nous présentons les participants du premiers au derniers selon le score FNC.
\begin{center}
 \begin{tabular}{ r | c c c c }
  \multicolumn{5}{c}{solat}\\
  Mesure     & agree & disagree & discuss & unrelated \\
  \hline
  Précision & 0.59  & 0.01     & 0.76    & 0.99      \\
  Rappel     & 0.51  & 0.33     & 0.75    & 0.97      \\
  F1score    & 0.55  & 0.03     & 0.76    & 0.98      \\
  \hline
  \hline
  Exactitude & \multicolumn{4}{c}{89.03}\\
  Score FNC & \multicolumn{4}{c}{7652.75}\\
  Pourcentage FNC & \multicolumn{4}{c}{81.99}\\
 \end{tabular}
 \captionof{table}{}
\end{center}

Bien que Solat in the Swen ait remporté la 1er place de la FNC, l'exactitudes et les scores ci-dessus ne sont pas les maximaux.

%%%%%%%%%%%%%%%%%%%%%%%%%%%%%%%%%%%%%%%%%%%%%%%%%%%%%%%%%%%%%%%%%%%%%%%


\subsubsection{Les sous-modèles de Solat in the Swen.}
\paragraph{Sous-modèle arborescent de Solat in the Swen}
Les sous-modèles de Solat n'ont pas eu de soumission à la FNC mais ils sont quand même intéressants à étudier car ils expliquent les résultats et les biais du modèles.

\begin{center}
 \begin{tabular}{ r | c c c c | c }
  \multicolumn{6}{c}{solat tree}\\
  \hline
            & agree & disagree & discuss & unrelated & Somme \\
  \hline
  agree     & 807   & 0        & 690     & 71        & 1568  \\
  disagree  & 147   & 1        & 334     & 68        & 550   \\
  discuss   & 500   & 1        & 2902    & 147       & 3550  \\
  unrelated & 16    & 0        & 160     & 14486     & 14662 \\
  \hline
  Somme     & 1470  & 2        & 4086    & 14772     & 20330 \\
 \end{tabular}
 \captionof{table}{}
\end{center}



On voit d'où le modèle principal tient son aversion du label \textbf{disagree}.

\begin{center}
 \begin{tabular}{ r | c c c c }
  \multicolumn{5}{c}{solat tree}\\
  Mesure     & agree & disagree & discuss & unrelated \\
  \hline
  Précision & 0.51  & 0.0      & 0.82    & 0.99      \\
  Rappel     & 0.55  & 0.5      & 0.71    & 0.98      \\
  F1score    & 0.53  & 0.0      & 0.76    & 0.98      \\
  \hline
  \hline
  Exactitude & \multicolumn{4}{c}{89.5}\\
  Score FNC & \multicolumn{4}{c}{7749.5}\\
  Pourcentage FNC & \multicolumn{4}{c}{83.03}\\
 \end{tabular}
 \captionof{table}{}
\end{center}


Ce sous-modèles ne tient pas vraiment compte de la classe \textbf{disagree} mais il a paradoxalement tout de même les plus hauts scores du challenge. En effet si Solat avait proposé uniquement ce modèle il aurait pris encore plus  d'avance par rapport aux autres participants.
%%%%%%%%%%%%%%%%%%%%%%%%%%%%%%%%%%%%%%%%%%
\paragraph{Sous-modèle de Deep Learning de Solat in the Swen}
\begin{center}
 \begin{tabular}{ r | c c c c | c }
  \multicolumn{6}{c}{solat deep}\\
  \hline
            & agree & disagree & discuss & unrelated & Somme \\
  \hline
  agree     & 911   & 115      & 143     & 399       & 1568  \\
  disagree  & 271   & 62       & 41      & 176       & 550   \\
  discuss   & 1396  & 137      & 1397    & 620       & 3550  \\
  unrelated & 2321  & 449      & 766     & 11126     & 14662 \\
  \hline
  Somme     & 4899  & 763      & 2347    & 12321     & 20330 \\
 \end{tabular}
 \captionof{table}{}
\end{center}



Nous voyons clairement que ce sous-modèle vient équilibrer le sous-modèle arborescent. Sa table de confusion montre une tentative d'uniformisation de la classe \textbf{disagree}.
\begin{center}
 \begin{tabular}{ r | c c c c }
  \multicolumn{5}{c}{solat deep}\\
  Mesure     & agree & disagree & discuss & unrelated \\
  \hline
  Précision & 0.58  & 0.11     & 0.39    & 0.76      \\
  Rappel     & 0.19  & 0.08     & 0.6     & 0.9       \\
  F1score    & 0.28  & 0.09     & 0.47    & 0.82      \\
  \hline
  \hline
  Exactitude & \multicolumn{4}{c}{66.38}\\
  Score FNC & \multicolumn{4}{c}{5677.25}\\
  Pourcentage FNC & \multicolumn{4}{c}{60.83}\\
 \end{tabular}
 \captionof{table}{}
\end{center}


Cette uniformisation a pour conséquence que ce sous-modèle est le pire de tous les modèles. Combiné avec le sous-modèle arborescent, il permet d'avoir une augmentation minime de la classe \textbf{disagree} au détriment des autres classes.
%%%%%%%%%%%%%%%%%%%%%%%%%%%%%%%%%%%%%%%%%%%%%%%%%%%%%%%%%%%


\subsection{Le système Athene}


\begin{center}
 \begin{tabular}{ r | c c c c | c }
  \multicolumn{6}{c}{athene}\\
  \hline
            & agree & disagree & discuss & unrelated & Somme \\
  \hline
  agree     & 709   & 57       & 669     & 133       & 1568  \\
  disagree  & 193   & 56       & 193     & 108       & 550   \\
  discuss   & 388   & 30       & 2853    & 279       & 3550  \\
  unrelated & 16    & 3        & 86      & 14557     & 14662 \\
  \hline
  Somme     & 1306  & 146      & 3801    & 15077     & 20330 \\
 \end{tabular}
 \captionof{table}{}
\end{center}



La table de confusion d'Athene ressembe beaucoup à celle de Solat. Ce modèle tente beaucoup plus de labéliser des titres d'articles comme \textbf{diagree}. Mais il a beaucoup plus de mal à distinguer la classe \textbf{discuss} de la classe \textbf{agree}.

\begin{center}
 \begin{tabular}{ r | c c c c }
  \multicolumn{5}{c}{athene}\\
  Mesure     & agree & disagree & discuss & unrelated \\
  \hline
  Précision & 0.45  & 0.1      & 0.8     & 0.99      \\
  Rappel     & 0.54  & 0.38     & 0.75    & 0.97      \\
  F1score    & 0.49  & 0.16     & 0.78    & 0.98      \\
  \hline
  \hline
  Exactitude & \multicolumn{4}{c}{89.4}\\
  Score FNC & \multicolumn{4}{c}{7639.75}\\
  Pourcentage FNC & \multicolumn{4}{c}{81.85}\\
 \end{tabular}
 \captionof{table}{}
\end{center}


Athene a la meilleure exactitude de toute la FNC car elle classe très bien les \textbf{unrelated} (qui ne valent pas beaucoup de points dans le score FNC).


\subsection{UCL Machine Reader}




\begin{center}
 \begin{tabular}{ r | c c c c | c }
  \multicolumn{6}{c}{uclmr}\\
  \hline
            & agree & disagree & discuss & unrelated & Somme \\
  \hline
  agree     & 697   & 7        & 770     & 94        & 1568  \\
  disagree  & 144   & 37       & 277     & 92        & 550   \\
  discuss   & 424   & 35       & 2882    & 209       & 3550  \\
  unrelated & 44    & 2        & 261     & 14355     & 14662 \\
  \hline
  Somme     & 1309  & 81       & 4190    & 14750     & 20330 \\
 \end{tabular}
 \captionof{table}{}
\end{center}


Nous observons une tentative d'uniformisation proportionnelle de la table de confusion.
UCLMR sait très bien classé les \textbf{unrelated}. Néanmoins, il manque apparemment de traits pour distinguer les \textbf{agree} des \textbf{discuss}.


\begin{center}
 \begin{tabular}{ r | c c c c }
  \multicolumn{5}{c}{uclmr}\\
  Mesure     & agree & disagree & discuss & unrelated \\
  \hline
  Précision & 0.44  & 0.07     & 0.81    & 0.98      \\
  Rappel     & 0.53  & 0.46     & 0.69    & 0.97      \\
  F1score    & 0.48  & 0.12     & 0.74    & 0.98      \\
  \hline
  \hline
  Exactitude & \multicolumn{4}{c}{88.4}\\
  Score FNC & \multicolumn{4}{c}{7619.0}\\
  Pourcentage FNC & \multicolumn{4}{c}{81.63}\\
 \end{tabular}
 \captionof{table}{}
\end{center}




\end{document}
\grid
