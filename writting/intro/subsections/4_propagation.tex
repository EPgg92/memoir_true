
\subsubsection{Les réseaux sociaux : médiations des Fakes News}
Selon \href{https://hal.inria.fr/hal-01281190}{Maksym Gabielkov et al, 2016}, 59\% des liens partagés sur Twitter n'ont jamais été cliqués.
En d'autres termes, la plupart des gens semblent retweeter des nouvelles sans jamais les lire.
Les personnes qui partagent sans lire des articles propagent certainement des Fake News sans le savoir.

Pour évaluer une Fake News nous n'avons pas besoin de dîplome.
En effet des études ont montré que les déterminismes sociaux n'étaient pas suffisants pour prévoir le partage de Fake News.
L'éducation, le sexe, l'âge, etc. ne sont pas des critères distinctifs.
\textcolor{red}{(Retrouver études...)}
Nous voyons alors que personne est à l'abri des Fake News. Seuls la pensée critique et le recul nous permettraient d'être protégés des Fake News.


\subsubsection{Le biais de confirmation.}
Les biais de confirmation sont un aspect déroutant de la pensée humaine.
Nous pourrions penser que l'homme ait acquis une pensée analytique développée pour arriver à ce niveau d'intelligence.
Et pourtant il est soumis au biais de confirmation.
Ce biais cognitif consiste à privilégier les informations confirmant nos idées préconçues ou nos hypothèses.
De plus il nous fait aussi négliger les informations jouant en défaveur de nos conceptions.

Ainsi les personnes tenantes de la thèse \og des extraterrestres gouvernent le pays\fg sont plus enclin à croire la thèse \og des reptiliens au pouvoir \fg que la thèse \og Barack Obama est un être humain \fg.

Les Fake News relayant souvent des informations conspirationnistes, il est facile pour un tenant de croire celle-ci plutôt que de croire des versions officielles.
Le biais de confirmation agit dans tous les sujets confondus. Il n'est pas uniquement cantonné au conspirationnisme. Les Fake News utilisent ce biais de manière idéologique pour renforcer nos croyances et nous rendre son information plausible.

\subsubsection{Tri de l'information selon Kahneman.}
La thèse centrale de Daniel Kahneman, dans son livre \og Système 1 / Système 2 : Les deux vitesses \fg, est qu'il y a une dichotomie entre deux modes de pensée.
Le système 1 est rapide, instinctif et émotionnel, alors que le système 2 est plus lent, plus réfléchi et plus logique.
Il définit les biais cognitifs associés à chacun de ces modes de pensée.
Il montre que l'on donne une trop grande importance au jugement humain.

Selon Kahneman nous nous reposons plus souvent sur le système 1.
Ce qui expliquerait notre partage de Fake News quand nous sommes émotionnellement impliqués.
