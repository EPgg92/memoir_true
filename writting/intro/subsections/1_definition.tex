\subsubsection{Produire des Fake News; une intention de tromper.}
%Distinguer les fake news des informations erronées dues à une erreur
%scientifique ou journalistique.

Un énoncé peut être vrai ou faux.
Il est vrai : s'il est en adéquation avec le monde.
Il est faux : s'il ne correspond pas à la réalité.
Les nouvelles (news) sont des énoncés.
Est-ce que l'énoncé \og J'aime les tartes. \fg  est une news ?
Non, toutes les news sont des énoncés; mais seul certains énoncés sont des news. Des journalistes ont fait une liste non exhaustive des critères d'une news (\href{http://www.tandfonline.com/doi/full/10.1080/1461670X.2016.1150193}{Tony Harcup et Deirdre O’Neill, 2016}).
La news est une histoire d'une personne de pouvoir ou célèbre.
Elle est parfois une bonne ou une mauvaise nouvelle.
Souvent elle est une surprise ou un phénomène qui concerne beaucoup de personnes.
Certaines sont juste des suivis, des divertissements ou des faits divers.
En somme, retenons qu'une nouvelle est une histoire en lien avec la réalité.
Ce qui nous intéresse c'est sa véracité selon l'interprétation du monde réel.
Cependant, leurs sujets sont variés, mais ce n'est pas pertinent pour nous.
De manière, plus générale une nouvelle est une histoire relayée par un média.

L'erreur est une opinion, un jugement ou une information non conforme avec la réalité ou la vérité.
L'erreur est inconsciente.
Elle n'est pas faite exprès.
Celle-ci démasquée, elle tend à être corrigée.
Une nouvelle erronée est donc une histoire médiatisée qui n'est pas vraie.
Elle ne correspond pas à la vérité.
La nouvelle erronée est due à une erreur scientifique ou bien une erreur journalistique.
C'est-à-dire une mauvaise manipulation de l'information par l'un de ces deux corps.
Les médias précautionneux corrigent leurs nouvelles erronées par des articles de démenti.
Ceci devrait être le code de déontologie des journalistes pour l'honnêteté intellectuelle.
Une Fake News\footnote{En français, nouvelle truquée mais le terme Fake News est devenu tellement commun
en français qu'il sera utilisé dans ce mémoire.} n'est pas une nouvelle erronée.
Cependant elles sont souvent confondues.

Parfois des nouvelles sont volontairement erronées.
On dit alors qu'elles sont fausses.
Il faut distinguer deux types de fausses nouvelles.
Le point important ici est l'intention derrière.

Les fausses nouvelles dans un but bien-veillant s'appellent satire ou information parodique.
Ces parodies imitent les médias.
Ils volent même jusqu'à leur nom parfois (le Gorafi est l'anagramme du Figaro).
Mais au lieu de diffuser de vraies informations, les parodies proposent un contenu décalé, sarcastique ou canularesque.
Le but premier de ce genre de nouvelle est le divertissement.
Même si l'on trompe le lecteur, on ne cherche pas à lui nuire mais plutôt à le faire rire.
Enfin il y a toujours des exceptions où l'information semble tellement crédible qu'elle est ensuite utilisée comme source fiable.
Les parodies donnent des informations délibérément fausses.
La tromperie est totalement assumée et même souvent revendiquée.

Les fausses nouvelles dans le but de volontairement tromper sont ce que l'on appelle Fake News.
Elle se distingue de l'erreur car elle n'est pas le produit du hasard ou d'une mauvaise manipulation.
Et elle se distingue de la satire et de la parodie car elle n'est ni assumée ni revendiquée fausse.
La Fake News provient d'un ensemble de médias.
Elle participe à la désinformation via les médias et les réseaux sociaux.
Souvent, les Fake News sont écrites par des anonymes difficilement contestables et calomniables.

\subsubsection{Limites de la définition.}
Certes une erreur peut arriver dans le traitement ou la création de l'information.
Mais ces erreurs sont-elles légitimes ?
La science progresse en faisant des erreurs.
D'ailleurs, dans la définition de l'étude scientifique l'erreur joue un rôle important.
L'étude scientifique est la recherche perpétuelle de l'erreur.
À défaut de nous apprendre ce qu'est la vérité, la science nous montre ce qu'elle n'est pas.
Donc l'erreur scientifique est légitime au bon fonctionnement de la science.
Peut-on en dire autant de l'erreur journalistique ?
Le journaliste doit faire un compte rendu exhaustif, objectif et vraisemblable du sujet qu'il traite.
Si une erreur non scientifique vient se faufiler dans son article c'est qu'il a mal fait son travail.

Par exemple, on voit toujours des médias relayer l'information que les vaccins causent l'autisme chez l'enfant.
Cette croyance persiste après une publication d'Andrew Wakefield
\footnote{Andrew Wakefield est un ancien chirurgien britannique et chercheur médical connu pour ses prétentions frauduleuses entre le vacin ROR et l'autisme.
Il fut radié de l'ordre des médecins en mai 2010 pour défaut à son devoir de consultant responsable.}.
Cette publication fut démentie des centaines de fois par des autorités compétentes, mais les médias de grande audience relayent toujours ce message de méfiance vaccinale.
Le devoir du journaliste n'est alors pas respecter .
Les médias traditionnels entretiennent le discours erroné des antivax
\footnote{Partisans de la controverse sur la vaccination qui remet en cause son efficacité et la sécurité de certains vaccins.}.
Celui-ci fait s'écouler beaucoup d'encre.
Mais cette controverse est de la désinformation pure et d'une grande malhonnêteté intellectuelle.
