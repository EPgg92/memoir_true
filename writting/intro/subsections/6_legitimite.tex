Comme nous avons pu le voir dans les sous-sections précédentes.
Les Fake News sont de fausses nouvelles qui cherchent à tromper volontairement.
Elles participent pour la majorité à une pollution du web dûe au système monétaire de la publicité en ligne.
Elles servent parfois un propos idéologique avec de mauvais argument.
Ceci les rend moins crédibles pour ceux qui découvrent le pot aux roses.
Elles sont produites par des militants anonymes anti-intellectuels prônant la désinformation sur de sombres forums.
Elles inondent les réseaux sociaux d'inepties et sont partagées en masse sans être lues.
Elles véhiculent des titres avec de fausses idées alors qu'un simple coup d'oeil au corps de l'article rend le propos infondé.
De plus,Elles coutent la vie à certains.
Et pour finir elles peuvent devenir un instrument politique redoutable et autoritaire.

Pour toutes ces raisons il est légitime de vouloir combattre le phénomène des Fake News.
