
\subsubsection{Comment vérifier des informations ?}
%Exposer différentes méthodes pour les différents types de médias.
Il y a deux moyens de détecter une Fake News.
Par soi-même en cherchant les indices du canular.
Ou en utilisant un moteur de recherche de canular sur un domaine internet.

Premièrement voyons de manière non-exhaustive quelques techniques pour détecter une Fake News:
\begin{description}
 \item [Avant de partager] il faut se questionner sur ce qu'il est raconté dans l'article et vérifier les sources. On est responsable de ce que l'on partage.
 \item [Est-ce une information ?] Il faut se poser différentes questions: Est-ce cela à un intérêt public ? Est-ce factuel ? Est-ce vérifié ? Cela permet de distinguer les avis et les rumeurs des informations.
 \item [Ce site est-il fiable ?] A-t-il une page \og À propos \fg ? Est-il parodique ? Quelles sont les sources de ce site ?
\end{description}
Beaucoup de techniques spécifiques pour chaque média existent ! Nous ne pouvons pas être exhaustif ici.

Les Fake News ont fait apparaître de nouveaux sites spécialisés dans la détection de canulars.
Par exemple, en français, il existe le \href{http://www.lemonde.fr/verification/}{Décodex} propulsé par le journal le Monde. Ce site répertorie les autres sites selon leurs fiabilités.
En anglais il existe le site internet Polifacts de vérification des faits, qui vérifie la véracité des promesses et engagements pris par les politiques américains
%\paragraph{Les limites des MDR vérificateur de faits.}
%Premièrement passer des menottes au lieu d'ouvrir le débat n'est pas une bonne solution !

%Deuxièmement les conflits d'intérêts des grands groupes qui font du fact-checking n'est pas à négliger. Une source de connaissance univoque n'est pas objective.
\subsubsection{Méthode informatique: Stance detection.}
Cette tâche peut aussi être résolue avec un succès modeste par l'apprentissage automatique.
En effet une des réponses au phénomène des Fake News est l'apparition de réseaux neuronaux complexes qui permettent de manière partielle de repérer les Fake News. Ce repérage passe par une détection des partis pris (stance detection).

Ainsi dans la deuxième partie de ce mémoire, nous allons voir plus en détail ce qu'est la stance detection.
Au travers de tâches partagées, nous allons découvrir les techniques pour l'utiliser.

Dans la troisième partie nous nous essayerons à la Stance detection aussi.
Et donnerons une analyse comparative de nos résultats par rapport à l'état de l'art.
