\subsubsection{Le contexte des Fake News.}
%Dans une presse de plus en plus virtualisée chacun peut devenir l'auteur d'un article.
L'invention qui a le plus contribué à l'essor des médias est certainement l'imprimerie.
Elle nous fait passer des histoires orales à la presse.
L'information est figée sur du papier.
Elle n'est plus perdue ou transformée par le locuteur de l'histoire.
À partir de ces documents, on peut faire des versions officielles et approuvées par une autorité.
L'information fut pendant très longtemps partagée de manière verticale.
La source d'autorité de la connaissance était plus ou moins légitime et compétente.
La connaissance était donnée par les médias et leur vision. Il n'y avait pas de sources contestataires.

Puis fut créé internet ! Internet permet un partage des connaissances horizontal.
Tout le monde disposant d'une connexion au réseau peut participer à la connaissance générale.
Chacun peut écrire, partager et relayer des informations.
Les médias se sont développé sur internet et surtout ils s'y sont multipliés.
La pluralité du web qui permet de mieux recroiser ses sources.
Nous ne sommes plus dans la vision dogmatique des grands groupes qui possèdent les chaînes télévisuelles.

%La liberté d'expression d'internet permet certes de diffuser la connaissance mais aussi des inepties.
Nous avons acquis une énorme liberté d'expression avec l'avènement d'internet comme média souverain de la pluralité.
Mais ce nouveau régime pluriel a aussi des désavantages.
La démocratie de la connaissance permet aux profanes de s'exprimer sur des sujets de pointe.
Ainsi, nombreux sont les opinions et les préjugés qui sont travestis en pseudo-vérités.
La liberté d'expression et la facilité d'accès à internet participent grandement à la désinformation globale.

%Notre comportement face à internet = Knowing more and understanding less in the age of big data. (The internet of us Michael Patrick Lynch)
De plus la multiplicité qu'engendre internet pose un problème à notre compréhension du monde.
En effet comme le dit Michael Lynch, internet c'est \og connaître plus et comprendre moins [...]\fg
\footnote{Citation : \textit{Knowing more and understanding less in the age of big data} sous-titre du livre \textit{The internet of us}, de Michael Lynch ISBN-10: \href{https://www.kirkusreviews.com/book-reviews/michael-patrick-lynch/the-internet-of-us/}{0871406616}}.
M. Lynch conteste la notion largement acceptée qu'internet est un avantage parce qu'il rend plus d'informations disponibles à plus de gens plus rapidement et facilement.
\subsubsection{Le système monétaire des Fake News.}


%Explication rapide du clickbait.
Le clickbait désigne un contenu web qui vise à attirer le maximum de passages d'internautes afin de générer des revenus publicitaires en ligne.
Le clickbait affiche des gros titres racoleurs.
Il est souvent mensonger et sensationnel.
L'exactitude et les sources de l'article sont inexistantes.
Le but du clikbait est d'être partagé massivement sur les réseaux sociaux.

%Comment le clickbait est utilisé dans la presse actuelle ?
Maintenant que la presse virtuelle est multiple, pour être rentable, elle doit générer une offre publicitaire non négligeable.
Le clickbait est un effet pervers d'internet.
Les possesseurs de contenu peuvent gagner de l'argent par le biais de la publicité.
Cela crée des nouveaux systèmes monétaires.

%Les Fake news le premier client du clickbait.
Les Fake News et le clickbait associés répondent à la crise profonde de la presse papier au profit des réseaux sociaux comme médias.
De plus les Fake News alimentent une méfiance envers les médias traditionnels.
Elles font croire avec leurs \og faits alternatifs\fg qu'on tente de cacher quelque chose à la population.

\subsubsection{Raisons idéologiques.}
%Instrumentalisation de l'information pour servir un propos politique.
Les raisons idéologiques de produire des Fake News ne manquent pas.
En particulier en politique, où elles sont utilisées à tout va.
%Exemple Pizzagate Hillary Clinton
Un exemple probant est le Pizzagate.
En résumé le Pizzagate est une \og théorie\fg conspirationniste prétendant qu'il existe un réseau de pédophilie autour de John Podesta, l'ancien directeur de campagne d'Hillary Clinton.
Cette histoire fut rapidement démentie par les services de police et une majorité des médias américains. Mais les conséquences ne sont pas négligeables.
Un fusillade, heureusement sans blessé, a eu lieu dans la pizzeria où étaient soi-disant séquestrés les enfants du réseau pédophile de Podesta.
