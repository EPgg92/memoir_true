\documentclass[onecolumn, 12pt]{article}

  \usepackage[utf8]{inputenc}
  \usepackage[T1]{fontenc}
  \usepackage[francais]{babel}
  \usepackage{layout}
  \usepackage{verbatim}[4]

  \title{La détection automatique de parti-pris au service de la vérification
  des faits.}
  \author{Enzo Poggio}
\begin{document}
\maketitle{}

%%%%%%%%%%%%%%%%%%%%%%%%%%%%%%%%%%%%%%%%%%%%%%%%%%%%%%%%%%%%%%%%%%
%%%%%%%% Part 1 Introduction au problème %%%%%%%%%%%%%%%%%%%%%%%%%
%%%%%%%%%%%%%%%%%%%%%%%%%%%%%%%%%%%%%%%%%%%%%%%%%%%%%%%%%%%%%%%%%%


\part{Le phénomène des Fake News.}

\section{Qu'est-ce qu'une Fake News?}
  \subsection{Définition d'une Fake News.}
    \subsubsection{Produire des Fake News; une intention de tromper.}
    Distinguer les fake news des informations éronées dues à une erreur
    scientifique ou journalistique.

    Distinguer la fake news de la satire.

    Définir la Fake News par son intentionnalité de nuire.
    \subsubsection{Limites de la définition}
    Une nouvelle qui cause du tort en étant fausse involontairement ne
    doit-elle pas être condidérée comme une fake news quand même ? Revenir
    sur l'erreur journalistique involontaire. Parler de la responsabilité
    journalistique par rapport à la communication de l'information.

    Exemple la vaccination/antivaxxed.

  \subsection{Pourquoi faire des fakes news?}
    \subsubsection{Le contexte des fake news.}
    Distinction manquante en Français Fake vs False news. Revenir sur la
    distinction entre un canular et fausse nouvelle.

    Dans une presse de plus en plus virtualisée chacun peut devenir l'auteur
    d'un article.

    La liberté d'expression d'internet permet certes de diffuser la connaissance
    mais aussi des inepties.

    Notre comportement face à intenet = Knowing more and understanding less in
    the age of big data. (The internet of us Michael Patrick Lynch)

    \subsubsection{Le système monértaire des fakes news.}
    Explication rapide du clickbait.

    Comment le clickbait est utilisé dans la presse actuelle ?

    Les Fake news le premier client du clickbait.

    \subsubsection{Raisons politiques.}
    Instrumentilisation de l'informations pour servir un propos politique.

    Exemple Pizzagate Hillary Clinton

  \subsection{Les origines des Fake News.}
    \subsubsection{Des médias négligeants.}
    La négligence et l'unicité des médias fait que les Fake News peuvent se
    propager facilement. Certains journalistes ne font pas le travail de
    vérifier leurs sources.

    \subsubsection{Des organisations malveillantes internationnales.}
    Expliquer comment 4Chan et Reddit sont des acteurs premiers dans la création
    de Fake news.

  \subsection{La propagation des Fake News.}
    \subsubsection{Les réseaux sociaux: médiations des Fakes News}
    Montrer comment les utilisateurs des réseaux sociaux partage sans lire des
    fausses informations.

    Montrer qu'il est difficile de distinguer une Fake News d'une News en dépit
    de nos déterminismes sociaux.

    \subsubsection{Le biais de confirmation.}
    Expliquer comment les biais de confirmations facilite le partage de fake
    news.

    \subsubsection{Tri de l'information selon Kahneman.}
    Expliquer les biais du au deux systèmes concurrents dans notre cerveau.

  \subsection{Les risques des Fake News.}

    \subsubsection{Les risques politiques.}
    Comment les fakes news peuvent aider à cacher des scandales politiques.

    Exemple les alternatives facts de l'institution de Donald Trump.

    Les fakes news peuvent être institutionnalisés.

    Exemple les climato-sceptiques.
    \subsubsection{Les risques sanitaires.}
    Comment la santé peut être mise en danger par de fausse informations?
    \subsubsection{Détournement des news vers des sujets clivants.}
    Les fake news peuvent être colportées par des personnes convaicus de leur
    véracité.(Les platistes qui pensent que la terre est vraiment plate; que la
    NASA et Hollywood ont fait un film de marche sur la lune).

    Les fakes news détourne l'attention de la population vers des problèmes
    factices et souvent résolus depuis des années.

    Le plus gros danger des fake news c'est d'y croire.
  \subsection{Légimité du projet et motivations.}
    Synthétiser les différents propos précédents et montrer à quel point il est
    important et urgent de s'attaquer au problèmes des fakes news.

  \subsection{Comment détecter une fakenews?}

    \subsubsection{Comment vérifier des informations?}
    Exposer différentes méthodes pour les différents types de médias.

    Des décodeurs du web (le monde) ou bien de Polifacts...
    \subsubsection{Les limites des MDR vérificateur de faits.}
    Premièrement passer des menottes au lieu d'ouvrir le débat n'est pas une
    bonne solution !

    Deuxièmement les conflits d'intèrêts des grands groupes qui font du
    fact-checking n'est pas à négliger. une source de connaissance univoque
    n'est pas objective.
    \subsubsection{Méthode informatique: Détection de parti pris
    ( ou Stance detection).}
    Expliquer que cette tâche peut être effectuée par un ordinateur grâce à la
    stance détection.

%%%%%%%%%%%%%%%%%%%%%%%%%%%%%%%%%%%%%%%%%%%%%%%%%%%%%%%%%%%%%%%%%%
%%%%%%%% Part 2 état de l'art %%%%%%%%%%%%%%%%%%%%%%%%%%%%%%%%%%%%
%%%%%%%%%%%%%%%%%%%%%%%%%%%%%%%%%%%%%%%%%%%%%%%%%%%%%%%%%%%%%%%%%%
\part{État de l'art}
.
\section{Vers une définition.}
.
  \subsection{Formalisation de la détection de parti.}
.
  \subsection{Domaine de la détection de parti.}
.
  \subsection{La genèse de la détection de parti pris.}
.
  \subsection{Application générale et relative au \textit{Fake News}.}
.
\section{\textit{SemEval-2016 Task 6}.}
.
  \subsection{Description générale de la tâche.}
.
    \subsubsection{Sous-tâche A.}
.
    \subsubsection{Sous-tâche B.}
.
    \subsubsection{Évaluation.}
.
  \subsection{Les participants.}
.
    \subsubsection{Soumissions particulièrement intéressantes.}
.
\section{\textit{Fake News Challenge}}
.
  \subsection{Description générale de la tâche}
.
    \subsubsection{But}
.
    \subsubsection{Organisation générale.}
.
    \subsubsection{Donneés et origines des données.}
.
    \subsubsection{L'évaluation.}
.
    \subsubsection{Baseline.}
.
    \subsubsection{Les participants}
.
  \subsection{Solat in the swen.}
.
    \subsubsection{Méthodologie.}
.
    \subsubsection{Traits et modèles utilisés.}
.
    \subsubsection{Algorithmes utilisés.}
.
    \subsubsection{Résultat et Analyses.}
.
  \subsection{UCL Machine reading.}
.
    \subsubsection{Méthodologie.}
.
    \subsubsection{Traits et modèles utilisés.}
.
    \subsubsection{Algorithmes utilisés.}
    .
    \subsubsection{Résultat et Analyses.}
    .
  \subsection{Athene (UKP Lab).}
    .
    \subsubsection{Méthodologie.}
    .
    \subsubsection{Traits et modèles utilisés.}
    .
    \subsubsection{Algorithmes utilisés.}
    .
    \subsubsection{Résultat et Analyses.}
    .
    \subsection{Autres participants.}
    .
    \subsection{Discussion comparative des articles}
    .
%%%%%%%% Section SemEval %%%%%%%%%%%%%%%%%%%%%%%%%%%%%%%%%%%%

%%%%%%%%%%%%%%%%%%%%%%%%%%%%%%%%%%%%%%%%%%%%%%%%%%%%%%%%%%%%%%%%%%
%%%%%%%% Part 3 Contribution original %%%%%%%%%%%%%%%%%%%%%%%%%%%%
%%%%%%%%%%%%%%%%%%%%%%%%%%%%%%%%%%%%%%%%%%%%%%%%%%%%%%%%%%%%%%%%%%

\part{Propositions d'améliorations.}
\section{Nom de mon sytème.}
Nous allons tenter ici d'apporter une améliorations aux systèmes déjà connu de
classification de fake News.

\subsection{Premier système (baseline).}
\subsubsection{Modèle}
Description exhaustive du modèle que l'on envisage de créer.
\subsubsection{Données.}
Expliquer que l'on va réutiliser les données de la FNC.

Expliquer que pour des tests systèmes on utilisera peut-être des corpus de
données décrit dans l'état de l'art.
\subsubsection{Méthodologie.}
Si il y a une méthodologie particulière adoptée, quel est-elle ?
\subsubsection{Traits choisis.}
Quels sont mes traits utilisés?
\subsubsection{Algorithmes utilisés.}
Quel algorithme j'utilise?
\subsubsection{Résultat et Analyses.}
Quels sont les résultats obtenus?

\subsection{Deuxième système (upgrade).}
  \subsubsection{Modèle amélioré.}
  Description exhaustive du modèle amélioré.
  \subsubsection{Changement?}
  Réécrire ici chacunes des sous-partie qui auraient subit un changement entre
  la baseline et l'upgrade.

\subsection{Troisième système (final).}
\subsubsection{Modèle final.}
Description exhaustive du modèle amélioré.
\subsubsection{Changement?}
Réécrire ici chacunes des sous-partie qui auraient subit un changement entre
l'upgrade et le final.
  \subsubsection{Résultats et Analyses finals.}
  Quels sont les résultats obtenus?
  \subsubsection{Limites.}
  Exposer les éventuelles limites méthologiques et d'analyses.
\end{document}
\grid
