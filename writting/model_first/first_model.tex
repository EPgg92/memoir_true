\documentclass[onecolumn, 12pt]{article}

%!TeX spellcheck =fr
  \usepackage[utf8]{inputenc}
  \usepackage{hyperref}
  \usepackage[T1]{fontenc}
  \usepackage[francais]{babel}
  \usepackage{layout}
  \usepackage{verbatim}[4]
  \usepackage{color}

  \usepackage{graphicx}
  \usepackage{caption}
  \usepackage{subcaption}

  \usepackage{titlesec}

  \setcounter{secnumdepth}{4}

  \titleformat{\paragraph}
  {\normalfont\normalsize\bfseries}{\theparagraph}{1em}{}
  \titlespacing*{\paragraph}
  {0pt}{3.25ex plus 1ex minus .2ex}{1.5ex plus .2ex}


\title{Premier modèle pour répondre à la FNC.}
  \author{Enzo Poggio}
\begin{document}
\maketitle{}


\begin{abstract}
 Présentation du premier modèle, son architecture, ses résultats et son analyse.
 Nous utilisons l'ensemble de test final.
\end{abstract}

\section{Premier modèle.}
\subsection{Explication du modèle.}
Comme nous l'avons vu précédement les combinaisons entre les modèles déjà présentés à la FNC peuvent atteindre de hauts résultats.
Notre premier modèle sera un jet naif de combinaisons de deux modèles.
C'est-à-dire que nous prendrons deux sous-modèles qui génèreront des probalités de label pour chaque entrée.
Nous nous s'inspirerons de la méthode d'unification de Solat.
Nous utiliserons donc aussi un \textit{ 50/50 weighted average} pour combiner les résultats.

Notre première architecture utilisera le sous-modèle arborescent de Solat et le modèle de UCL Machine Reader.

\textcolor{red}{Notre seconde architecture utilisera le sous-modèle arborescent de Solat et le modèle d'Athene. (\textbf{Je n'arrive pas à trouver comment générer les probabilités de classe dans le code d'Athene. Donc je n'ai pas de second résultats pour le moment.})}
\subsection{Résultats}

\subsubsection{Modéles mixte Solat/UCLMR}

\begin{center}
 \begin{tabular}{ r | c c c c | c }
  \multicolumn{6}{c}{mix solat uclmr}\\
  \hline
            & agree & disagree & discuss & unrelated & Somme \\
  \hline
  agree     & 963   & 0        & 819     & 121       & 1903  \\
  disagree  & 198   & 1        & 377     & 121       & 697   \\
  discuss   & 606   & 3        & 3612    & 243       & 4464  \\
  unrelated & 21    & 0        & 166     & 18162     & 18349 \\
  \hline
  Somme     & 1788  & 4        & 4974    & 18647     & 25413 \\
 \end{tabular}
 \captionof{table}{}
\end{center}
Cette table de confusion ressemble beaucoup à celle de Solat dans l'analyse sur 80\% du corpus.
Sans surprise, la classe disagree n'est pas améliorée.
Par contre, les classes agree, discuss et unrelated sont légèrement augmentées.

\begin{center}
 \begin{tabular}{ r | c c c c }
  \multicolumn{5}{c}{mix solat uclmr}\\
  Mesure     & agree & disagree & discuss & unrelated \\
  \hline
  Précision & 0.51  & 0.0      & 0.81    & 0.99      \\
  Rappel     & 0.54  & 0.25     & 0.73    & 0.97      \\
  F1score    & 0.52  & 0.0      & 0.77    & 0.98      \\
  \hline
  \hline
  Exactitude & \multicolumn{4}{c}{89.47}\\
  Score FNC & \multicolumn{4}{c}{9617.25}\\
  Pourcentage FNC & \multicolumn{4}{c}{82.54}\\
 \end{tabular}
 \captionof{table}{}
\end{center}
Nous constatons un gain de 0.52\% par rapport au meilleur des systèmes de la FNC.
Ceci est peu mais montre une amélioration minime de la précision de la classe agree et de la classe discuss.
La classe disagree est encore négligée par le modèle arborescent de Solat.
En conséquence, le modèle UCLMR biaisé par Solat ne permet pas d'améliorer la précision de la classe disagree.

\end{document}
\grid
